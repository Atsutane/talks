%% Licensed as Creative Commons 3.0 BY SA
%% Authors: Thorsten 'Atsutane' Toepper

\documentclass[mode=print,paper=screen,size=10pt,style=horatio]{powerdot}
\usepackage[utf8x]{inputenc}
\usepackage{amsmath}
\usepackage{amsfonts}
\usepackage{amssymb}
\usepackage{color}
\usepackage{graphicx}
\usepackage{ngerman}
\usepackage{url}
\usepackage{listings}

\newcommand{\Anf}[1]{\glqq #1\grqq}

\lstset{
	basicstyle=\footnotesize\ttfamily,
	numbers=left,
	numberstyle=\tiny,
	xleftmargin=10pt,
	numbersep=10pt,
}

\author{Atsutane}
\title{Arch Linux - Eine einfache, schlanke Linux Distribution}

\pdsetup{palette=default}
\begin{document}

\maketitle
\begin{slide}{Inhalt}
  \tableofcontents[content=sections]
\end{slide}

%% Ein Slide zur Vorstellung des Autors
%% Licensed as Creative Commons 3.0 BY SA
%% Authors: Thorsten 'Atsutane' Toepper

\begin{slide}{\"{U}ber den Autor}
	\begin{itemize}
		\item{Seit XYZ ArchLinux Benutzer}
		\item{Zwei oder drei weitere Punkte mit Bezug zum Thema}
	\end{itemize}
\end{slide}


%% Wie auch in den Wikis eine Vorstellung der Distribution 
\section{Was ist Arch Linux?}
%% Licensed as Creative Commons 3.0 BY SA
%% Authors: Thorsten 'Atsutane' Toepper

\begin{slide}{Was ist Arch Linux?}
	\begin{itemize}
		\item{Unabhängig entwickelte Community Distribution}
		\item{i686 und x86\_64 optimiert}
		\item{basiert auf:
			\begin{itemize}
				\item{dem Rolling Release Modell}
				\item{Binärpaketen}
				\item{pacman als Paketverwaltung}
				\item{dem Ports ähnelndem ABS}
			\end{itemize}
		}
		\item{Die Entwicklung fokussiert auf:
			\begin{itemize}
				\item{Minimalismus}
				\item{Eleganz}
				\item{korrekten Code}
				\item{Aktualität}
			\end{itemize}
		}
	\end{itemize}
\end{slide}

\begin{slide}{Vorteile}
	\begin{itemize}
		\item{Arch Linux ist ...
			\begin{itemize}
				\item{leichtgewichtig}
				\item{flexibel}
				\item{einfach}
				\item{darauf aus, möglichst \textit{UNIX}-artig zu sein}
			\end{itemize}
		}
		\item{minimale Installation, das System ist in beliebige Richtungen zu führen.}
	\end{itemize}
\end{slide}


%% Kurze Erläuterung Pacmans
\section{Pacman}
%% Licensed as Creative Commons 3.0 BY SA
%% Authors: Thorsten 'Atsutane' Toepper

\begin{slide}{Allgemeines}
	\begin{itemize}
		\item{pacman ist in C geschrieben}
		\item{Konfigurationsdateien
			\begin{itemize}
				\item{/etc/pacman.conf}
				\item{/etc/pacman.d/mirrorlist}
			\end{itemize}
		}
		\item{Die standardm"{a}ßig aktiven Repositories sind:
			\begin{itemize}
				\item{core}
				\item{extra}
				\item{community}
			\end{itemize}
		}
		\item{Daneben verf"{u}gbar sind:
			\begin{itemize}
				\item{testing}
				\item{community-testing}
			\end{itemize}
		}
	\end{itemize}
\end{slide}

\begin{slide}{Verwendung}
	\begin{itemize}
		\item{pacman $<$Schalter$>$ [Optionen] $<$Pakete$>$}
		\item{Wichtige Schalter sind:
			\begin{itemize}
				\item{-Q: Abfragen der Paketdatenbank mit den installierten
					Paketen.}
				\item{-S: Abfragen der Paketdatenbank und installieren von
					Paketen.}
				\item{-R: Entfernen von Paketen.}
				\item{-U: Installieren lokaler Pakete.}
			\end{itemize}
		}
	\end{itemize}
\end{slide}

\begin{slide}{Beispiele}
	\begin{itemize}
		\item{pacman -S $<$Paket$>$}
		\item{pacman -Ss $<$Suchbegriff$>$
			\footnote{Optionen auch mit -Q verf"{u}gbar}
		}
		\item{pacman -Si $<$Paket$>$¹}
		\item{pacman -Syu
			\begin{itemize}
				\item{\textit{y} synchronisiert die Datenbank mit dem Mirror}
				\item{\textit{u} aktualisiert die installierten Pakete}
			\end{itemize}
		}
		\item{pacman -Rscn $<$Paket$>$
			\begin{itemize}
				\item{\textit{s} entfernt Pakete von denen nur dieses Paket
					abh"{a}ngt}
				\item{\textit{c} entfernt Pakete die von diesem abh"{a}ngen}
				\item{\textit{n} entfernt auch die Konfigurationsdateien}
			\end{itemize}
		}
	\end{itemize}
\end{slide}

\begin{slide}{Paketformat}
	\begin{itemize}
		\item{name-version-release-architektur.pkg.tar.gz
			\begin{itemize}
				\item{gzip komprimierter Tarball}
			\end{itemize}
		}
		\item{.PKGINFO
			\begin{itemize}
				\item{pkgname}
				\item{pkgver}
				\item{pkgdesc}
				\item{url}
				\item{builddate}
				\item{packager}
				\item{size}
				\item{arch}
				\item{license}
			\end{itemize}
		}		
	\end{itemize}
\end{slide}

%% Das Slide soll nur kurz behandelt werden
\begin{slide}{Mitgelieferte Skripte}
		\begin{itemize}
			\item{makepkg}
			\item{rankmirrors}
			\item{repo-add / repo-remove}
			\item{vercmp}
			\item{testpkg}
			\item{pkgdelta}
			\item{testdb}
			\item{pacman-optimize}
		\end{itemize}
\end{slide}



%% Eine genauere Betrachtung des ABS, PKGBUILDs und des AURs
\section{ABS und AUR}
%% Licensed as Creative Commons 3.0 BY SA
%% Authors: Thorsten 'Atsutane' Toepper

\begin{slide}{Arch Build System}
	\begin{itemize}
		\item{Ports "{a}hnlichees System}
		\item{Was kann ich als User damit anfangen?
			\begin{itemize}
				\item{Erstellen von Paketen.}
				\item{Unkompliziertes anpassen von Paketen aus den 
					Repositories.}
			\end{itemize}
		}
		\item{Aufgeteilt in:
			\begin{itemize}
				\item{abs}
				\item{makepkg}
				\item{pacman}
				\item{PKGBUILDs}
				\item{AUR - Arch User Repository}
			\end{itemize}
		}
	\end{itemize}
\end{slide}

%% Konfiguration und Struktur von abs
\begin{slide}{abs}
	\begin{itemize}
		\item{Konfiguration
			\begin{itemize}
				\item{/etc/abs.conf}
				\item{ABSROOT -  Standard: /var/abs}
				\item{SYNCHSERVER - Standard: rsync.archlinux.org}
				\item{REPOS - Standard: Alle, auch nicht geführte}
			\end{itemize}
		}
		\item{Struktur
			\begin{itemize}
				\item{ABSROOT
					\begin{itemize}
						\item{Repository
							\begin{itemize}
								\item{Paket}
							\end{itemize}
						}
					\end{itemize}
				}
				\item{/var/abs/core/kernel26/
					\begin{itemize}
						\item{PKGBUILD config config.x86\_64 kernel26.install
							kernel26.preset}
					\end{itemize}
				}
			\end{itemize}
		}
	\end{itemize}
\end{slide}

%% Grobes auseinandernehmen eines PKGBUILDs
\begin{slide}{PKGBUILD}
	\begin{itemize}
		\item{}
	\end{itemize}
\end{slide}

%% Konfiguration, besondere Optionen
\begin{slide}{makepkg}
	\begin{itemize}
		\item{}
	\end{itemize}
\end{slide}

%% Screenshot, wird gesprochen erklärt
\begin{slide}{Arch User Repository}
	\begin{itemize}
		\item{}
	\end{itemize}
\end{slide}



%% Ein kurzer Blick auf unseren Umgang mit Einstellungen, Kernelmodulen
%% und DAEMONs. Kurz: rc.conf und Initsystem
\section{Systemkonfiguration und Initsystem}
%% Licensed as Creative Commons 3.0 BY SA
%% Authors: Thorsten 'Atsutane' Toepper

\begin{slide}{Aufbau der Systemkonfiguration}
	\begin{itemize}
		\item{BSD artig zentral \"{u}ber eine Datei
			\begin{itemize}
				\item{\textit{/etc/rc.conf}
					\lstinputlisting{samples/rc_conf.sample}
				}
			\end{itemize}
		}

	\end{itemize}
\end{slide}


\begin{slide}{Stufen des Initsystems}
	\begin{tabular}{|c|c|}
		\hline Level & Funktion\\
		\hline 0 & Halt\\
		\hline 1 & Single-User\\
		\hline 2 & Nicht benutzt\\
		\hline 3 & Multi-User (Standard)\\
		\hline 4 & Nicht benutzt\\
		\hline 5 & X11\\
		\hline 6 & Reboot\\
		\hline
	\end{tabular}
\end{slide}


%% Eine Erklärung des Trusted User Systems mit Gegenüberstellung
%% zu den Devs um den Aufbau der Community zu erklären
\section{Trusted User System}
%% Licensed as Creative Commons 3.0 BY SA
%% Authors: Thorsten 'Atsutane' Toepper

\begin{slide}{Aufgaben der Trusted User}
	\begin{itemize}
		\item{Verwaltung des:
			\begin{itemize}
				\item{community Repositories}
				\item{AURs}
			\end{itemize}
		}
		\item{Das abstimmen in den Entscheidungen bez\"{u}glich des AURs und des community Repositories.}
	\end{itemize}
\end{slide}

\begin{slide}{Wie wird man Trusted User?}
	\begin{itemize}
		\item{F\"{u}r die Bewerbung n\"{o}tig:
			\begin{itemize}
				\item{Ein Trusted User als Sponsor}
				\item{Eine gewisse Bekanntheit bei den Trusted Usern}
				\item{Einige Pakete im AUR zur F\"{a}higkeitsdarlegung}
			\end{itemize}
		}
		\item{Entscheidung f\"{a}llt nach den Regeln der Trusted User Bylaws\footnote{http://aur.archlinux.org/trusted-user/TUbylaws.html}:
			\begin{itemize}
				\item{5 Tage Diskussion auf der aur-general Mailing Liste}
				\item{7 Tage Abstimmungszeitraum f\"{u}r die TUs.}
				\item{66\% der TUs m\"{u}ssen abstimmen.}
				\item{Nach 3 Monaten darf man es erneut versuchen.}
			\end{itemize}
		}
	\end{itemize}
\end{slide}

\begin{slide}{Unterschied zwischen Trusted User und Developer}
	\begin{itemize}
		\item{Developer sind f\"{u}r das Kernsystem zust\"{a}ndig:
			\begin{itemize}
				\item{Repositories: core, extra, testing}
				\item{best\"{a}tigen der Funktionalit\"{a}t von Paketen aus testing, welche nach core sollen.}
				\item{treffen der wegweisenden Entscheidungen}
			\end{itemize}
		}
	\end{itemize}
\end{slide}



%% Eine minimale Beschreibung einer Installation auf welcher
%% aufbauend dargestellt werden kann in welche Richtung man
%% mit einem Arch System gehen kann (Server, Desktop, Laptop).
\section{Installation}
%% Licensed as Creative Commons 3.0 BY SA
%% Authors: Thorsten 'Atsutane' Toepper

\begin{slide}{Installationsmedien}
	\begin{itemize}
		\item{Images:
			\begin{itemize}
				\item{i686 / x86\_64 core Image}
				\item{i686 / x86\_64 netinstall Image}
			\end{itemize}
		}
		
		\item{Je f\"{u}r:
			\begin{itemize}
				\item{USB-Stick}
				\item{CD-ROM}
			\end{itemize}
		}
		\item{Zu finden unter:
			\begin{itemize}
				\item{\url{http://www.archlinux.org/download/}}
			\end{itemize}
		}
	\end{itemize}
\end{slide}

\begin{slide}{Installation}
	\begin{itemize}
		\item{AIF - Arch Installation Framework
			\begin{itemize}
				\item{Textinstaller}
				\item{Skriptbar $\Rightarrow{}$ automatische Installationen m\"{o}glich}
			\end{itemize}
		}
		\item{\"{U}bliche Installation:
			\begin{itemize}
				\item{115 Pakete}
				\item{500 MB}
			\end{itemize}
		}
	\end{itemize}
\end{slide}



%% Ansprache der Unterteilung der Community in regionaler
%% und anderer Hinsicht und Nennung einiger zur Community
%% gehörender Projekte (bspw. Chakra)
\section{Community}
%% Licensed as Creative Commons 3.0 BY SA
%% Authors: Thorsten 'Atsutane' Toepper

%% Ansprache der Unterteilung der Community in regionaler
%% und anderer Hinsicht und Nennung einiger zur Community
%% geh\"{o}render Projekte (bspw. Chakra)

\begin{slide}{Aufteilung der Community}
	\begin{itemize}
		\item{Internationale Kerncommunity:
			\begin{itemize}
				\item{\url{http://www.archlinux.org/}}
			\end{itemize}
		}
		\item{Deutschsprachige Community:
			\begin{itemize}
				\item{\url{http://www.archlinux.de/}}
			\end{itemize}
		}
		\item{Weitere Communities:
			\begin{itemize}
				\item{\url{http://www.archlinux.org/moreforums/}}
			\end{itemize}
		}
	\end{itemize}
\end{slide}

\begin{slide}{Arch bezogene Projekte}
	\begin{itemize}
		\item{pacman - Paketverwaltung
			\begin{itemize}
				\item{\url{http://projects.archlinux.org/pacman.git/}}
			\end{itemize}
		}
		\item{Chakra - auf Arch basierende KDEMod Livedistribution
			\begin{itemize}
				\item{\url{http://www.chakra-project.org/}}
			\end{itemize}
		}
	\end{itemize}
\end{slide}



\end{document}
