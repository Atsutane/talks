%% Licensed as Creative Commons 3.0 BY SA
%% Authors: Thorsten 'Atsutane' Toepper

\documentclass[mode=print,paper=screen,size=10pt,style=horatio]{powerdot}
\usepackage[utf8x]{inputenc}
\usepackage{color}
\usepackage{graphicx}
\usepackage{ngerman}
\usepackage{url}
\usepackage{listings}

\newcommand{\Anf}[1]{\glqq #1\grqq}

\author{Atsutane}
\title{Arch Linux - Eine einfache, schlanke Linux Distribution}

\pdsetup{palette=default}
\begin{document}

\maketitle
\begin{slide}{Inhalt}
  \tableofcontents[content=sections]
\end{slide}

%% Ein Slide zur Vorstellung des Autors
%% Licensed as Creative Commons 3.0 BY SA
%% Authors: Thorsten 'Atsutane' Toepper

\begin{slide}{\"{U}ber den Autor}
	\begin{itemize}
		\item{Seit XYZ ArchLinux Benutzer}
		\item{Zwei oder drei weitere Punkte mit Bezug zum Thema}
	\end{itemize}
\end{slide}


%% Wie auch in den Wikis eine Vorstellung der Distribution 
\section{Was ist Arch Linux?}
%% Licensed as Creative Commons 3.0 BY SA
%% Authors: Thorsten 'Atsutane' Toepper

\begin{slide}{Was ist Arch Linux?}
	\begin{itemize}
		\item{Unabhängig entwickelte Community Distribution}
		\item{i686 und x86\_64 optimiert}
		\item{basiert auf:
			\begin{itemize}
				\item{dem Rolling Release Modell}
				\item{Binärpaketen}
				\item{pacman als Paketverwaltung}
				\item{dem Ports ähnelndem ABS}
			\end{itemize}
		}
		\item{Die Entwicklung fokussiert auf:
			\begin{itemize}
				\item{Minimalismus}
				\item{Eleganz}
				\item{korrekten Code}
				\item{Aktualität}
			\end{itemize}
		}
	\end{itemize}
\end{slide}

\begin{slide}{Vorteile}
	\begin{itemize}
		\item{Arch Linux ist ...
			\begin{itemize}
				\item{leichtgewichtig}
				\item{flexibel}
				\item{einfach}
				\item{darauf aus, möglichst \textit{UNIX}-artig zu sein}
			\end{itemize}
		}
		\item{minimale Installation, das System ist in beliebige Richtungen zu führen.}
	\end{itemize}
\end{slide}


%% Kurze Erläuterung Pacmans
\section{Pacman}


%% Eine genauere Betrachtung des ABS, PKGBUILDs und des AURs
\section{ABS und AUR}


%% Eine Erklärung des Trusted User Systems mit Gegenüberstellung
%% zu den Devs um den Aufbau der Community zu erklären
\section{Trusted User System}


%% Eine minimale Beschreibung einer Installation auf welcher
%% aufbauend dargestellt werden kann in welche Richtung man
%% mit einem Arch System gehen kann (Server, Desktop, Laptop).
\section{Installation}


%% Ansprache der Unterteilung der Community in regionaler
%% und anderer Hinsicht und Nennung einiger zur Community
%% gehörender Projekte (bspw. Chakra)
\section{Community}


\end{document}
