%% Licensed as Creative Commons 3.0 BY SA
%% Authors: Thorsten 'Atsutane' Toepper

\begin{slide}{Allgemeines}
	\begin{itemize}
		\item{pacman ist in C geschrieben}
		\item{Die standardm"{a}ßig aktiven Repositories sind:
			\begin{itemize}
				\item{core}
				\item{extra}
				\item{community}
			\end{itemize}
		}
		\item{Daneben verf"{u}gbar sind:
			\begin{itemize}
				\item{testing}
				\item{community-testing}
			\end{itemize}
		}
	\end{itemize}
\end{slide}

\begin{slide}{Verwendung}
	\begin{itemize}
		\item{pacman $<$Schalter$>$ [Optionen] $<$Pakete$>$}
		\item{Wichtige Schalter sind:
			\begin{itemize}
				\item{-Q: Abfragen der Paketdatenbank mit den installierten
					Paketen.}
				\item{-S: Abfragen der Paketdatenbank und installieren von
					Paketen.}
				\item{-R: Entfernen von Paketen.}
				\item{-U: Installieren lokaler Pakete.}
			\end{itemize}
		}
	\end{itemize}
\end{slide}

\begin{slide}{Beispiele}
	\begin{itemize}
		\item{pacman -S $<$Paket$>$}
		\item{pacman -Ss $<$Suchbegriff$>$
			\footnote{Optionen auch mit -Q verf"{u}gbar}
		}
		\item{pacman -Si $<$Paket$>$¹}
		\item{pacman -Syu
			\begin{itemize}
				\item{\textit{y} synchronisiert die Datenbank mit dem Mirror}
				\item{\textit{u} aktualisiert die installierten Pakete}
			\end{itemize}
		}
		\item{pacman -Rscn $<$Paket$>$
			\begin{itemize}
				\item{\textit{s} entfernt Pakete von denen nur dieses Paket
					abh"{a}ngt}
				\item{\textit{c} entfernt Pakete die von diesem abh"{a}ngen}
				\item{\textit{n} entfernt auch die Konfigurationsdateien}
			\end{itemize}
		}
	\end{itemize}
\end{slide}

