%% Licensed as Creative Commons 3.0 BY SA
%% Authors: Thorsten 'Atsutane' Toepper

\begin{slide}{Aufgaben der Trusted User}
	\begin{itemize}
		\item{Verwaltung des:
			\begin{itemize}
				\item{community Repositories}
				\item{AURs}
			\end{itemize}
		}
		\item{Das abstimmen in den Entscheidungen bez\"{u}glich des AURs und des community Repositories.}
	\end{itemize}
\end{slide}

\begin{slide}{Wie wird man Trusted User?}
	\begin{itemize}
		\item{F\"{u}r die Bewerbung n\"{o}tig:
			\begin{itemize}
				\item{Ein Trusted User als Sponsor}
				\item{Eine gewisse Bekanntheit bei den Trusted Usern}
				\item{Einige Pakete im AUR zur F\"{a}higkeitsdarlegung}
			\end{itemize}
		}
		\item{Entscheidung f\"{a}llt nach den Regeln der Trusted User Bylaws\footnote{http://aur.archlinux.org/trusted-user/TUbylaws.html}:
			\begin{itemize}
				\item{5 Tage Diskussion auf der aur-general Mailing Liste}
				\item{7 Tage Abstimmungszeitraum f\"{u}r die TUs.}
				\item{66\% der TUs m\"{u}ssen abstimmen.}
				\item{Nach 3 Monaten darf man es erneut versuchen.}
			\end{itemize}
		}
	\end{itemize}
\end{slide}

\begin{slide}{Unterschied zwischen Trusted User und Developer}
	\begin{itemize}
		\item{Developer sind f\"{u}r das Kernsystem zust\"{a}ndig:
			\begin{itemize}
				\item{Repositories: core, extra, testing}
				\item{best\"{a}tigen der Funktionalit\"{a}t von Paketen aus testing, welche nach core sollen.}
				\item{treffen der wegweisenden Entscheidungen}
			\end{itemize}
		}
	\end{itemize}
\end{slide}

