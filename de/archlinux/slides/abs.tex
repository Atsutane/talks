%% Licensed as Creative Commons 3.0 BY SA
%% Authors: Thorsten 'Atsutane' Toepper

\begin{slide}{Arch Build System}
	\begin{itemize}
		\item{Ports \"{a}hnliches System}
		\item{Was kann ich als User damit anfangen?
			\begin{itemize}
				\item{Erstellen von Paketen.}
				\item{Unkompliziertes anpassen von Paketen aus den 
					Repositories.}
			\end{itemize}
		}
		\item{Aufgeteilt in:
			\begin{itemize}
				\item{abs}
				\item{makepkg}
				\item{pacman}
				\item{PKGBUILDs}
				\item{AUR - Arch User Repository}
			\end{itemize}
		}
	\end{itemize}
\end{slide}

%% Konfiguration und Struktur von abs
\begin{slide}{abs}
	\begin{itemize}
		\item{Konfiguration
			\begin{itemize}
				\item{/etc/abs.conf}
				\item{ABSROOT -  Standard: /var/abs}
				\item{SYNCHSERVER - Standard: rsync.archlinux.org}
				\item{REPOS - Standard: Alle, auch nicht gef\"{u}hrte}
			\end{itemize}
		}
		\item{Struktur
			\begin{itemize}
				\item{ABSROOT
					\begin{itemize}
						\item{Repository
							\begin{itemize}
								\item{Paket}
							\end{itemize}
						}
					\end{itemize}
				}
				\item{/var/abs/core/kernel26/
					\begin{itemize}
						\item{PKGBUILD config config.x86\_64 kernel26.install
							kernel26.preset}
					\end{itemize}
				}
			\end{itemize}
		}
	\end{itemize}
\end{slide}

%% Grobes auseinandernehmen eines PKGBUILDs
\begin{slide}{PKGBUILD - Werte}
	\lstinputlisting[]{samples/PKGBUILD_pt1.sample}
\end{slide}
\begin{slide}{PKGBUILD - Funktionen}
	\lstinputlisting[]{samples/PKGBUILD_pt2.sample}
\end{slide}


%% Konfiguration, besondere Optionen
\begin{slide}{makepkg}
	\begin{itemize}
		\item{Konfiguration: \textit{/etc/makepkg.conf}}
		\item{makepkg [Schalter]
			\begin{itemize}
				\item{\textit{c} Verzeichnis im Anschluss s\"{a}ubern}
				\item{\textit{C} Auch Quelldateien entfernen}
				\item{\textit{s} fehlende Abh\"{a}ngigkeiten installieren}
				\item{\textit{r} installierte Abh\"{a}ngigkeiten im Anschluss
					wieder entfernen}
				\item{\textit{i} Paket im Anschluss installieren}
				\item{\textit{g} Checksummen berechnen und ausgeben}
			\end{itemize}
		}
	\end{itemize}
\end{slide}

%% Screenshot, wird gesprochen erkl\"{a}rt
\begin{slide}{Arch User Repository}
	\begin{itemize}
		\item{}
	\end{itemize}
\end{slide}

