%% Licensed as Creative Commons 3.0 BY SA
%% Authors: Thorsten 'Atsutane' Toepper

\documentclass[mode=print,paper=screen,size=11pt,style=horatio]{powerdot}
\usepackage[utf8x]{inputenc}
\usepackage{amsmath}
\usepackage{amsfonts}
\usepackage{amssymb}
\usepackage{color}
\usepackage{graphicx}
\usepackage{ngerman}
\usepackage{url}
\usepackage{listings}

\newcommand{\Anf}[1]{\glqq #1\grqq}

\lstset{
	basicstyle=\footnotesize\ttfamily,
	numbers=left,
	numberstyle=\tiny,
	xleftmargin=10pt,
	numbersep=10pt,
}

\pddefinetemplate{basic}{
  titlehook=Bl,
  titlepos={.2\slidewidth,.91\slideheight},
  titlewidth=.77\slidewidth,
  titlefont=\color{white}\large\bfseries\raggedright,
  clockcolor=white,clockpos={.99\slidewidth,\slideheight},
  lfpos={.2\slidewidth,.04\slideheight},
  lffont=\tiny\color{white},
  rfpos={.97\slidewidth,.04\slideheight},
  rffont=\tiny\color{white},
  tocslidesep=.6ex,
  textheight=.68\slideheight,
  ifsetup=landscape,
    toctcolor=pdcolor1,
    tochlcolor=pdcolor1,
    tochltcolor=white,
    tocpos={.01\slidewidth,.81\slideheight},
    tocwidth=.14\slidewidth,
    tocsecm={\psline[linewidth=.5pt,linecolor=pdcolor1]%
      (-.05,-.05)(.143\slidewidth,-.05)},
  ifsetup=portrait,
    toctcolor=white,
    tochlcolor=white,
    tochltcolor=pdcolor1,
    tocsecsep=.6ex,
    stochook=tr,stocpos={.55\slidewidth,.09\slideheight},
    stocfont=\tiny\raggedleft,
    ntocpos={.57\slidewidth,.09\slideheight}
}{{%
  \psset{linecolor=pdcolor1}%
  \psframe*(0,0)(\slidewidth,.1\slideheight)%
  \psframe*(0,.85\slideheight)(\slidewidth,\slideheight)%
  \psset{linecolor=pdcolor2,linewidth=1pt}%
  \psline(.17\slidewidth,0)(.17\slidewidth,.1\slideheight)%
  \psline(.17\slidewidth,.85\slideheight)(.17\slidewidth,\slideheight)%
  \rput(1.2,.50){\color{white}\begin{tiny}Arch Linux\end{tiny}}
  \rput(6,.50){\color{white}\begin{tiny}Thorsten T\"{o}pper - Proseminar Hochschule Mannheim - 28. April 2010\end{tiny}}
}}

%% Symbols used in itemize sections
\def\labelitemi{\small$\blacktriangleright{}$}
\def\labelitemii{\small$\triangleright{}$}
\def\labelitemiii{\small$\bullet{}$}
\def\labelitemiv{\small\textdiamond{}}

\author{Thorsten T\"{o}pper}
\title{Arch Linux - Eine einfache, schlanke Linux Distribution}
\pdsetup{palette=default}

\begin{document}

\maketitle
\begin{slide}{Inhalt}
  \tableofcontents[content=sections]
\end{slide}

%% Wie auch in den Wikis eine Vorstellung der Distribution 
\section{Allgemein: Linux}
%% Licensed as Creative Commons 3.0 BY SA
%% Authors: Thorsten 'Atsutane' Toepper

\begin{slide}{Aufbau einer Linuxdistribution}
	\begin{itemize}
		\item{Linux-Kernel}
		\item{Paketverwaltung}
		\item{Shell}
		\item{X11}
		\item{Desktop Environment/Window Manager}
		\item{Weitere mitgelieferte Software}
	\end{itemize}
\end{slide}

\begin{slide}{Bekannte Distributionen}
	\begin{itemize}
		\item{\textbf{Debian}
			\begin{itemize}
				\item{Community}
				\item{Basis von
				\begin{itemize}
				 	\item{\textbf{Ubuntu} und Derivate}
					\item{\textbf{BackTrack}}
				\end{itemize}
				}
				\item{Aptitude Paketverwaltung}
			\end{itemize}
		}
		\item{\textbf{RedHat}
			\begin{itemize}
				\item{Firma}
				\item{Basis von
					\begin{itemize}
						\item{\textbf{Fedora} und Derivate}
					\end{itemize}
				}
				\item{RPM Paketverwaltung}
			\end{itemize}
		}
	\end{itemize}
\end {slide}


%% Licensed as Creative Commons 3.0 BY SA
%% Authors: Thorsten 'Atsutane' Toepper

\begin{slide}{Was ist Arch Linux?}
	\begin{itemize}
		\item{Unabh\"{a}ngig entwickelte Community Distribution}
		\item{i686 und x86\_64 optimiert}
		\item{basiert auf:
			\begin{itemize}
				\item{dem Rolling Release Modell}
				\item{Bin\"{a}rpaketen}
				\item{pacman als Paketverwaltung}
				\item{dem Ports \"{a}hnelndem ABS}
			\end{itemize}
		}
		\item{Die Entwicklung fokussiert auf:
			\begin{itemize}
				\item{Minimalismus}
				\item{Eleganz}
				\item{korrekten Code}
				\item{Aktualit\"{a}t}
			\end{itemize}
		}
	\end{itemize}
\end{slide}

\begin{slide}{Allgemeine Vorteile}
	\begin{itemize}
		\item{Arch Linux ist ...
			\begin{itemize}
				\item{leichtgewichtig}
				\item{flexibel}
				\item{einfach}
				\item{darauf aus, m\"{o}glichst \textit{UNIX}-artig zu sein}
			\end{itemize}
		}
		\item{minimale Installation, das System ist in beliebige
			Richtungen zu f\"{u}hren.}
	\end{itemize}
\end{slide}

\begin{slide}{Einfache Paketverwaltung}
	\begin{itemize}
		\item{Bestehend aus:
			\begin{itemize}
				\item{pacman - Paketmanager}
				\item{abs - ports \"{a}hnelndes Package Build System}
			\end{itemize}
		}
	\end{itemize}
\end{slide}

\begin{slide}{Aktualit\"{a}t}
	\begin{itemize}
		\item{aktuelle stabile Version der Software}
		\item{Vanilla Software
			\begin{itemize}
				\item{Patches werden nur zur Vermeidung von Fehlern
					bei Updates eingespielt.}
			\end{itemize}
		}
	\end{itemize}
\end{slide}

\begin{slide}{Schlichtheit}
	\begin{itemize}
		\item{Die Philosophie der Distribution}
		\item{Das Basissystem besteht aus:
			\begin{itemize}
				\item{Kernel}
				\item{Paketverwaltung}
				\item{GNU Toolchain}
				\item{Einigen weiteren Programmen}
			\end{itemize}
		}
		\item{Das init System orientiert sich an BSD
			\begin{itemize}
				\item{Systemkonfiguration geschieht \"{u}ber eine Datei:
					\textit{/etc/rc.conf}}
				\item{Konfiguration wird \"{u}ber einen Editor angepasst}
			\end{itemize}
		}
	\end{itemize}
\end{slide}



%% Kurze Erl\"{a}uterung Pacmans
\section{Pacman}
%% Licensed as Creative Commons 3.0 BY SA
%% Authors: Thorsten 'Atsutane' Toepper

\begin{slide}{Allgemeines}
	\begin{itemize}
		\item{pacman ist in C geschrieben}
		\item{Die standardm"{a}ßig aktiven Repositories sind:
			\begin{itemize}
				\item{core}
				\item{extra}
				\item{community}
			\end{itemize}
		}
		\item{Daneben verf"{u}gbar sind:
			\begin{itemize}
				\item{testing}
				\item{community-testing}
			\end{itemize}
		}
	\end{itemize}
\end{slide}

\begin{slide}{Verwendung}
	\begin{itemize}
		\item{pacman $<$Schalter$>$ [Optionen] $<$Pakete$>$}
		\item{Wichtige Schalter sind:
			\begin{itemize}
				\item{-Q: Abfragen der Paketdatenbank mit den installierten
					Paketen.}
				\item{-S: Abfragen der Paketdatenbank und installieren von
					Paketen.}
				\item{-R: Entfernen von Paketen.}
				\item{-U: Installieren lokaler Pakete.}
			\end{itemize}
		}
	\end{itemize}
\end{slide}

\begin{slide}{Beispiele}

\end{slide}



%% Eine genauere Betrachtung des ABS, PKGBUILDs und des AURs
\section{ABS und AUR}
%% Licensed as Creative Commons 3.0 BY SA
%% Authors: Thorsten 'Atsutane' Toepper

\begin{slide}{Arch Build System}
	\begin{itemize}
		\item{Ports "{a}hnlichees System}
		\item{Was kann ich als User damit anfangen?
			\begin{itemize}
				\item{Erstellen von Paketen.}
				\item{Unkompliziertes anpassen von Paketen aus den 
					Repositories.}
			\end{itemize}
		}
		\item{Aufgeteilt in:
			\begin{itemize}
				\item{abs}
				\item{makepkg}
				\item{pacman}
				\item{PKGBUILDs}
				\item{AUR - Arch User Repository}
			\end{itemize}
		}
	\end{itemize}
\end{slide}

%% Konfiguration und Struktur von abs
\begin{slide}{abs}
	\begin{itemize}
		\item{Konfiguration
			\begin{itemize}
				\item{/etc/abs.conf}
				\item{ABSROOT -  Standard: /var/abs}
				\item{SYNCHSERVER - Standard: rsync.archlinux.org}
				\item{REPOS - Standard: Alle, auch nicht geführte}
			\end{itemize}
		}
		\item{Struktur
			\begin{itemize}
				\item{ABSROOT
					\begin{itemize}
						\item{Repository
							\begin{itemize}
								\item{Paket}
							\end{itemize}
						}
					\end{itemize}
				}
				\item{/var/abs/core/kernel26/
					\begin{itemize}
						\item{PKGBUILD config config.x86\_64 kernel26.install
							kernel26.preset}
					\end{itemize}
				}
			\end{itemize}
		}
	\end{itemize}
\end{slide}

%% Grobes auseinandernehmen eines PKGBUILDs
\begin{slide}{PKGBUILD}
	\begin{itemize}
		\item{}
	\end{itemize}
\end{slide}

%% Konfiguration, besondere Optionen
\begin{slide}{makepkg}
	\begin{itemize}
		\item{}
	\end{itemize}
\end{slide}

%% Screenshot, wird gesprochen erklärt
\begin{slide}{Arch User Repository}
	\begin{itemize}
		\item{}
	\end{itemize}
\end{slide}



%% Ein kurzer Blick auf unseren Umgang mit Einstellungen, Kernelmodulen
%% und DAEMONs. Kurz: rc.conf und Initsystem
\section{Systemkonfiguration und Initsystem}
%% Licensed as Creative Commons 3.0 BY SA
%% Authors: Thorsten 'Atsutane' Toepper

\begin{slide}{Aufbau der Systemkonfiguration}
	\begin{itemize}
		\item{BSD artig zentral \"{u}ber eine Datei
			\begin{itemize}
				\item{\textit{/etc/rc.conf}
					\lstinputlisting{samples/rc_conf.sample}
				}
			\end{itemize}
		}

	\end{itemize}
\end{slide}


\begin{slide}{Stufen des Initsystems}
	\begin{tabular}{|c|c|}
		\hline Level & Funktion\\
		\hline 0 & Halt\\
		\hline 1 & Single-User\\
		\hline 2 & Nicht benutzt\\
		\hline 3 & Multi-User (Standard)\\
		\hline 4 & Nicht benutzt\\
		\hline 5 & X11\\
		\hline 6 & Reboot\\
		\hline
	\end{tabular}
\end{slide}


%% Eine Erkl\"{a}rung des Aufgabenbereiches von Devs und 
%% Trusted Usern, Gegen\"{u}berstellung um den Aufbau der
%% Community zu erkl\"{a}ren
\section{Developer \& Trusted User}
%% Licensed as Creative Commons 3.0 BY SA
%% Authors: Thorsten 'Atsutane' Toepper

\begin{slide}{Aufgaben der Developer}
	\begin{itemize}
		\item{Developer sind f\"{u}r das Kernsystem zust\"{a}ndig:
			\begin{itemize}
				\item{Repositories:
					\begin{itemize}
						\item{[core]}
						\item{[extra]}
						\item{[testing]}
					\end{itemize}
				}
				\item{[testing] Pakete pr\"{u}fen}
				\item{nach mehreren Sign-Offs nach [core] verschieben}
				\item{treffen der wegweisenden Entscheidungen}
			\end{itemize}
		}
	\end{itemize}
\end{slide}

\begin{slide}{Aufgaben der Trusted User}
	\begin{itemize}
		\item{Verwaltung des:
			\begin{itemize}
				\item{[community] Repositories}
				\item{AURs}
			\end{itemize}
		}
		\item{Entscheidungen treffen f\"{u}r:
			\begin{itemize}
				\item{AUR}
				\item{[community]}
				\item{neue Trusted User}
			\end{itemize}
		}
	\end{itemize}
\end{slide}

\begin{slide}{Wie wird man Trusted User?}
	\begin{itemize}
		\item{F\"{u}r die Bewerbung n\"{o}tig:
			\begin{itemize}
				\item{Trusted User als Sponsor}
				\item{gewisse Bekanntheit bei Trusted Usern}
				\item{F\"{a}higkeitsdarlegung mittels einiger Pakete im AUR}
			\end{itemize}
		}
		\item{Entscheidung nach Trusted User Bylaws:
			\begin{itemize}
				\item{5 Tage Diskussion auf [aur-general]}
				\item{7 Tage Abstimmungszeitraum}
				\item{66\% Abstimmungs-Beteiligung}
				\item{3 Monate Pause}
			\end{itemize}
		}
	\end{itemize}
\end{slide}



%% Eine minimale Beschreibung einer Installation auf welcher
%% aufbauend dargestellt werden kann in welche Richtung man
%% mit einem Arch System gehen kann (Server, Desktop, Laptop).
\section{Installation}
%% Licensed as Creative Commons 3.0 BY SA
%% Authors: Thorsten 'Atsutane' Toepper

\begin{slide}{Installationsmedien}
	\begin{itemize}
		\item{Images:
			\begin{itemize}
				\item{i686 / x86\_64 core Image}
				\item{i686 / x86\_64 netinstall Image}
			\end{itemize}
		}
		
		\item{Je f\"{u}r:
			\begin{itemize}
				\item{USB-Stick}
				\item{CD-ROM}
			\end{itemize}
		}
		\item{Zu finden unter:
			\begin{itemize}
				\item{\url{http://www.archlinux.org/download/}}
			\end{itemize}
		}
	\end{itemize}
\end{slide}

\begin{slide}{Installationsframework}
	\begin{itemize}
		\item{AIF - Arch Installation Framework
			\begin{itemize}
				\item{Textinstaller}
				\item{Skriptbar $\Rightarrow{}$ automatische Installationen m\"{o}glich}
			\end{itemize}
		}
	\end{itemize}
\end{slide}



%% Ansprache der Unterteilung der Community in regionaler
%% und anderer Hinsicht und Nennung einiger zur Community
%% geh\"{o}render Projekte (bspw. Chakra)
\section{Community}
%% Licensed as Creative Commons 3.0 BY SA
%% Authors: Thorsten 'Atsutane' Toepper

%% Ansprache der Unterteilung der Community in regionaler
%% und anderer Hinsicht und Nennung einiger zur Community
%% geh\"{o}render Projekte

\begin{slide}{Aufteilung der Community}
	\begin{itemize}
		\item{Internationale Community:
			\begin{itemize}
				\item{\url{http://www.archlinux.org/}}
			\end{itemize}
		}
		\item{Deutschsprachige Community:
			\begin{itemize}
				\item{\url{http://www.archlinux.de/}}
			\end{itemize}
		}
		\item{Weitere Communities:
			\begin{itemize}
				\item{\url{http://www.archlinux.org/moreforums/}}
			\end{itemize}
		}
	\end{itemize}
\end{slide}

\begin{slide}{Arch bezogene Projekte}
	\begin{itemize}
		\item{pacman - Paketverwaltung
			\begin{itemize}
				\item{\url{http://projects.archlinux.org/pacman.git/}}
			\end{itemize}
		}
		\item{Arch User Repository
			\begin{itemize}
				\item{\url{http://aur.archlinux.org/}}
			\end{itemize}
		}
		\item{AIF - Arch Installation Framework
			\begin{itemize}
				\item{\url{http://github.com/Dieterbe/aif/master/}}
			\end{itemize}
		}
	\end{itemize}
\end{slide}



\end{document}
