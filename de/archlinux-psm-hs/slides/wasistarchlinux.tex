%% Licensed as Creative Commons 3.0 BY SA
%% Authors: Thorsten 'Atsutane' Toepper

\begin{slide}{Arch Linux}
	\begin{itemize}
		\item{Unabh\"{a}ngige Community}
		\item{i686 und x86\_64 optimiert}
		\item{basiert auf:
			\begin{itemize}
				\item{Rolling Release Modell}
				\item{Bin\"{a}rpaketen}
				\item{pacman als Paketverwaltung}
			\end{itemize}
		}
		\item{Fokus auf:
			\begin{itemize}
				\item{Minimalismus}
				\item{Vanilla-Code}
				\item{Aktualit\"{a}t}
			\end{itemize}
		}
	\end{itemize}
\end{slide}

\begin{slide}{Allgemeine Vorteile}
	\begin{itemize}
		\item{Arch Linux ist ...
			\begin{itemize}
				\item{leichtgewichtig}
				\item{flexibel}
				\item{einfach}
			\end{itemize}
		}
		\item{minimale Installation}
		\item{abs gestaltet Paketierung einfach}
	\end{itemize}
\end{slide}

\begin{slide}{Allgemeine Nachteile}
	\begin{itemize}
		\item{Ank\"{u}ndigungen m\"{u}ssen verfolgt werden}
		\item{Probleme nach mehrmonatiger Aktualisierungs-Pause}
		\item{Fortgeschrittene Kenntnisse bzw. Interesse ist n\"{o}tig}
	\end{itemize}
\end{slide}

\begin{slide}{Schlichtheit}
	\begin{itemize}
		\item{KISS}
		\item{Basissystem besteht aus:
			\begin{itemize}
				\item{Kernel}
				\item{Paketverwaltung}
				\item{GNU Toolchain}
				\item{Texteditor}
			\end{itemize}
		}
		\item{BSD-artiges Initsystem}
		\item{Systemkonfiguration mittels einer Datei}
	\end{itemize}
\end{slide}

