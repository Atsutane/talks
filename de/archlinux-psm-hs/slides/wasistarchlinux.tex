%% Licensed as Creative Commons 3.0 BY SA
%% Authors: Thorsten 'Atsutane' Toepper

\begin{slide}{Was ist Arch Linux?}
	\begin{itemize}
		\item{Unabh\"{a}ngig entwickelte Community Distribution}
		\item{i686 und x86\_64 optimiert}
		\item{basiert auf:
			\begin{itemize}
				\item{dem Rolling Release Modell}
				\item{Bin\"{a}rpaketen}
				\item{pacman als Paketverwaltung}
				\item{dem Ports \"{a}hnelndem ABS}
			\end{itemize}
		}
		\item{Die Entwicklung fokussiert auf:
			\begin{itemize}
				\item{Minimalismus}
				\item{Eleganz}
				\item{korrekten Code}
				\item{Aktualit\"{a}t}
			\end{itemize}
		}
	\end{itemize}
\end{slide}

\begin{slide}{Allgemeine Vorteile}
	\begin{itemize}
		\item{Arch Linux ist ...
			\begin{itemize}
				\item{leichtgewichtig}
				\item{flexibel}
				\item{einfach}
				\item{darauf aus, m\"{o}glichst \textit{UNIX}-artig zu sein}
			\end{itemize}
		}
		\item{minimale Installation, das System ist in beliebige
			Richtungen zu f\"{u}hren.}
	\end{itemize}
\end{slide}

%% Wird wahrscheinlich entfernt, wird ja im Anschluss behandelt.
\begin{slide}{Einfache Paketverwaltung}
	\begin{itemize}
		\item{Bestehend aus:
			\begin{itemize}
				\item{pacman - Paketmanager}
				\item{abs - ports \"{a}hnelndes Package Build System}
			\end{itemize}
		}
	\end{itemize}
\end{slide}

\begin{slide}{Aktualit\"{a}t}
	\begin{itemize}
		\item{aktuelle stabile Version der Software}
		\item{Vanilla Software
			\begin{itemize}
				\item{Patches werden nur zur Vermeidung von Fehlern
					bei Updates eingespielt.}
			\end{itemize}
		}
	\end{itemize}
\end{slide}

\begin{slide}{Schlichtheit}
	\begin{itemize}
		\item{Die Philosophie der Distribution}
		\item{Das Basissystem besteht aus:
			\begin{itemize}
				\item{Kernel}
				\item{Paketverwaltung}
				\item{GNU Toolchain}
				\item{Einigen weiteren Programmen}
			\end{itemize}
		}
		\item{Das init System orientiert sich an BSD
			\begin{itemize}
				\item{Systemkonfiguration geschieht \"{u}ber eine Datei:
					\textit{/etc/rc.conf}}
				\item{Konfiguration wird \"{u}ber einen Editor angepasst}
			\end{itemize}
		}
	\end{itemize}
\end{slide}

