%% Licensed as Creative Commons 3.0 BY SA
%% Authors: Thorsten 'Atsutane' Toepper

\begin{slide}{Arch Build System}
	\begin{itemize}
		\item{Programme zu Paketierung von Software/Daten}
		\item{Verwendung zur:
			\begin{itemize}
				\item{Paketierung}
				\item{einfachen Modifikation der Bin\"{a}rpakete aus den Repositories}
			\end{itemize}
		}
		\item{Aufgeteilt in:
			\begin{itemize}
				\item{abs}
				\item{makepkg}
				\item{pacman}
				\item{PKGBUILDs}
				\item{AUR - Arch User Repository}
			\end{itemize}
		}
	\end{itemize}
\end{slide}


%% Grobe Erkl\"{a}rung eines PKGBUILDs
\begin{slide}{PKGBUILD}
	\begin{itemize}
		\item{einfache Textdatei}
		\item{\textit{Kuchenrezept} f\"{u}r ein Paket}
		\item{Festgelegt werden:
			\begin{itemize}
				\item{Paketname}
				\item{Version}
				\item{Revision dieser Version}
				\item{Quelle}
				\item{Lizenz}
				\item{Abh\"{a}ngigkeiten}
				\item{Funktion \texttt{build} zur Festlegung zu erledigender Schritte}
			\end{itemize}
		}
	\end{itemize}
\end{slide}

%% Konfiguration und Struktur von abs
\begin{slide}{abs}
	\begin{itemize}
		\item{l\"{a}d PKGBUILDs der Bin\"{a}rpakete}
		\item{synchronisiert die lokale Kopie mit dem Server}
	\end{itemize}
\end{slide}


%% Konfiguration, besondere Optionen
\begin{slide}{makepkg}
	\begin{itemize}
		\item{Shellskript}
		\item{\texttt{build} Funktion des PKGBUILD im aktuellen Verzeichnis}
		\item{makepkg erm\"{o}glicht:
			\begin{itemize}
				\item{anschließende Verzeichniss\"{a}uberung}
				\item{Installation fehlender Abh\"{a}ngigkeiten}
				\item{anschließendes entfernen dieser}
				\item{direkte Installation nach erfolgreicher Paketierung}
				\item{Berechnung von Pr\"{u}fsummen}
				\item{Erstellung von Sourceballs}
			\end{itemize}
		}
	\end{itemize}
\end{slide}

%% Screenshot, wird gesprochen erkl\"{a}rt
\begin{slide}{Arch User Repository}
	\begin{itemize}
		\item{Webseite: \url{http://aur.archlinux.org/}}
		\item{Benutzer k\"{o}nnen
			\begin{itemize}
				\item{eigene PKGBUILDs/Sourceballs hochladen}
				\item{anderer Benutzer Sourceballs herunterladen}
				\item{f\"{u}r Sourceballs stimmen}
				\item{verwaiste Pakete adoptieren}
			\end{itemize}
		}
	\end{itemize}
\end{slide}

